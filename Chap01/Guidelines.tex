%Formatting Guidelines for Writing Dissertation.
\chapter{Introduction}\label{Guidelines}
\section{Overview}
Remote patient monitoring (RPM) is a direct consequence of the increasing usage of technology in healthcare. It is the most effective means for a patient's health to be tracked without a need for frequent hospital visits. The doctors by using IoT (Internet of Things) devices such as sensors and microcontrollers are able to gather live data such as temperature, oxygen levels or heart rate of the patients who are at home or in far-off places. This is extremely valuable because it leads to the prevention of health issues as well as the provision of medical support on time by doctors.\\

On the other hand, when medical data is sent through the internet, it becomes necessary to guarantee security and privacy of such data. If there is any kind of unauthorized access or data leakage, the result would be the risk of the patient's safety. Thus, the security subsystem in the system should be not only strong in terms of protection but also be efficient enough to work on small, low-power IoT devices.\\

Our work is to introduce a secure remote patient monitoring system implementation that would support safety and efficiency of the communication process through the use of blockchain technology and Elliptic Curve Cryptography (ECC). The Arduino UNO and the ESP Wi-Fi module that are connected with sensors make up the first part of the system, which is used to collect patient data that is then uploaded to a cloud platform (ThingSpeak). After that, the cloud sends the data to the hospital server, which is the doctor's place, and they are able to remotely monitor the patient's condition through the data received. The usage of blockchain guarantees that only authorized users and devices have access to the data, whereas ECC is the reason for quick and simple data encryption in an IoT environment.

\section{Objectives}
The primary goals envisaged through this undertaking are:
\begin{itemize}
   \item To conceive and implement a remote patient monitoring system that gathers real-time health information through IoT sensors.
   \item To maintain data security and confidentiality by implementing a blockchain-enabled authentication method and using light cryptography (ECC).
   \item To establish good communication pathways between patient equipment, cloud and hospital server.
   \item To offer doctors an opportunity to remotely access patient medical records via a safe and user-friendly system.
   \item To trim down the computational load thereby making it possible for the system to function efficiently on hardware capable of limited resources such as Arduino and ESP modules.
\end{itemize}

\chapter{Literature Review}
\section{Overview}
Remote Patient Monitoring (RPM) is the vital element of the Internet of Medical Things (IoMT), a concept in which intelligent devices keep track of a patient's health continuously and the updated information is sent to the respective doctors or hospitals \cite{iot_healthcare_survey}. Different researchers throughout the years have come up with a variety of methods to ensure that such systems are not only secure but also dependable.

The first generation of systems were mainly concerned with gathering and transferring patients' data done through Wi-Fi or Bluetooth while the aspects of privacy and security were only partially taken care of \cite{thingspeak_rpm}. Consequently, healthcare data being highly sensitive even the slightest negligence of it would lead to a situation where data leakage, impersonation, or unauthorized access would occur.

\section{Review of Previous Works}

\subsection{Initial IoT-based Healthcare Systems}
Ancestor systems were implemented on basic IoT schematics consisting of sensors and cloud platforms (example: ThingSpeak or Blynk) \cite{thingspeak_rpm}. These frameworks were effective in monitoring vital signs; however, due to the absence of strict authentication and encryption measures, they were susceptible to hacking.

\subsection{Password-Based Authentication Schemes}
A number of later projects incorporated password-protected authentication to secure communications \cite{password_auth_weakness}. That said, the security provided by such methods was very weak because passwords can be easily guessed or stolen, thus leading to scenarios where the perpetrator can impersonate the victim or record/replay the communication.

\subsection{Public Key Cryptography (RSA) Based Approaches}
The next step was adoption heavy cryptographic schemes like RSA or alike to secure patient data \cite{rsa_iot_overhead}. The drawback of this approach was that these strong security measures demanded high computational power conflicting with the low-power nature of devices like Arduino or ESP modules.

\subsection{Lightweight Cryptography \& ECC-Based Methods}
Researcher solved this problem by switching to Elliptic Curve Cryptography (ECC), an algorithm achieving the same security level as RSA but with smaller key sizes and faster processing speed \cite{ecc_iot}. Presently, ECC is almost the standard for IoT applications, including healthcare.

\subsection{Blockchain-Enabled Systems}
Recent studies present medical data management with the help of blockchain technology \cite{blockchain_healthcare}. Blockchain functions as an incorruptible ledger, keeping track of data transactions and making them transparent, thus, trustworthy, and secure against hacker attacks. When combined with ECC, blockchain delivers a decentralized and lightweight authentication solution that suits real-time medical monitoring requirements \cite{blockchain_ecc_iot}.

\section{Research Gap}
The analyzed papers highlight that while the developed platforms facilitate data gathering and sharing, security is still an open issue and rarely is the trade-off between security and efficiency addressed. Some works call for significant computational resources, others are vulnerable to certain types of attacks. The key challenges identified include:

\begin{itemize}
   \item Need to operate under constrained energy of IoT devices,
   \item Implementation of lightweight cryptographic techniques to ensure security,
   \item Use of blockchain technology not only for security purposes but also to facilitate user authentication, and
   \item Being able to guarantee privacy and integrity of data during a continuous patient monitoring scenario.
\end{itemize}

\begin{table}[!ht]
    \caption{Comparative Analysis of Authentication Schemes in IoT-based Healthcare Systems} \vspace{0.2cm}
    \centering
    \resizebox{\textwidth}{!}{ 
    \begin{tabular}{|c|p{3.5cm}|p{3cm}|p{4.5cm}|}
        \hline
        \textbf{Year} & \textbf{Technique Used} & \textbf{Focus Area} & \textbf{Limitations} \\ \hline
        2018 & Three-factor authentication using passwords and biometrics & Wireless Sensor Networks in healthcare & High computation time; not suitable for low-power devices \\ \hline
        2019 & Lightweight ECC-based user authentication & Wearable IoT devices & Lacked decentralized data protection \\ \hline
        2020 & Password-based DRM authentication & IoT Applications & Weak against replay and impersonation attacks \\ \hline
        2021 & Three-factor IoT authentication using ECC & Healthcare Systems & Secure but lacked blockchain integration \\ \hline
        2022 & Secure three-factor authentication for IoT & IoT-based healthcare & Needed more scalability and efficiency \\ \hline
        2023 & Hyperelliptic curve-based IoD authentication & Internet of Drones & Good security but complex math operations \\ \hline
        2024 & PUF-based lightweight authentication & Internet of Drones & Focused on drones, not healthcare \\ \hline
        \textbf{Present Work (2025)} & \textbf{Blockchain + ECC-based lightweight authentication} & \textbf{Remote Patient Monitoring System} & \textbf{Addresses past issues; provides both efficiency and strong security} \\ \hline
    \end{tabular}
    }
    \label{tab:auth_schemes_comparison}
\end{table}

Table~\ref{tab:auth_schemes_comparison} presents a comparative analysis of various authentication schemes proposed for IoT-based healthcare systems over recent years. The table highlights the progression from basic password-based methods to more sophisticated approaches incorporating ECC and blockchain technology, culminating in the present work which addresses the limitations of previous approaches. 

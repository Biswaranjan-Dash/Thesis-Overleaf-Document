%Formatting Guidelines for Writing Dissertation.
\chapter{Introduction}\label{Guidelines}
\section{Introduction}
Forecasting is the backbone of several fields including energy management, financial markets, health care delivery, growth planning in rural areas and urban systems, as well as climate science. Better predictions enable stakeholders to foresee the future more clearly, spend their resources wisely, and make better decisions informed by data that will change business results as well as the societal good. Yet most conventional forecasting techniques, which deliver a single point estimate, can fail to account for one important factor — the uncertainty.
All predictive models carry inherent uncertainty, particularly when applied to real-world data. Point forecasts attempt to be an estimate of the future, but do not give a measure for the band of likely outcomes. As a result, predictions that don't match the observed values very well — large difference between predicted and actual behavior), can potentially lead decision makers astray to sub-optimal or risky decisions. This is especially noticeable in high-stakes areas such as predicting the weather, analyzing the stock market and resource management (where even slight errors of a forecast can lead to drastic losses or gains).
\par Probabilistic forecasting is a better alternative as it gives a range of possibilities by providing an estimate in the confidence interval in which the true value is most likely to fall. These measures help decision-makers to be ready for different situations, thus reducing risks of unpredictable events. But even though probabilistic forecasting has these clear benefits, it is currently underused in many machine learning applications. Secondly, the enormous challenge where there is no agreement over which is the best approach to use — a parametric or non-parametric approach — generally makes things even worse because neither of them provides high levels of precision and reliability across datasets.
Deep learning models are a new ensemble in the work of probabilistic forecasting. Nonetheless, there is limited literature treating the determination of-or one with a lead over another on the discriminative power of these techniques across different domains. For a particular distribution (e.g., Gaussian or Weibull) and benefit from simplicity but may not be suitable to the realities of the data. However, a class of non-parametric methods such as Quantile Regression (QR) based forecasting and Bootstrap-based forecasting do not assume any underlying distribution; but they are more flexible in general, which increases their complexity in terms of computations. As a result, approaches such as the Traditional and Advanced LUBE method appear increasingly viable for use in non-parametric forecasting; however, their accuracy may still suffer relative to more established methods.

\section{Objective}
This thesis aims to design and develop precise and dependable probabilistic forecasting methods using optimized deep learning models on quantitative time series data from diverse domains.
In particular, the objectives include:
\begin{itemize}
    \item Examination and comparative evaluation of various probabilistic forecasting approaches based on Deep Learning models across multiple datasets, including stock prices, electricity consumption, and web traffic.
    \item Development and evaluation of multiple hybrid probabilistic forecasting methods that integrate traditional approaches with DL models to enhance forecasting accuracy and applicability in real-world scenarios.
\end{itemize}



\section{Organization Of Thesis}
There are six chapters in this thesis. The six interconnected chapters are intended to
thoroughly examine Probabilistic Forecasting Methods using Deep Learning Models. From identifying the problem to developing and validating a solution, each chapter builds on the one before it to form a cohesive narrative that leads the reader through the research process. The chapters are
organized as follows:

\textbf{Chapter 1} comprises the background information and research motivation necessary to comprehend this thesis. 

\textbf{Chapter 2} contains a thorough Literature Review.

\textbf{Chapter 3} contains the six different parametric and non-parametric traditional probabilistic forecasting methods and their comparisons on five different datasets using four different evaluation metrics.

\textbf{Chapter 4} introduces the LUBE-Weibull based Hybrid Method.

\textbf{Chapter 5} introduces the LUBE-QR based Hybrid Method.

\textbf{Chapter 6} contains the conclusion of the thesis and highlights future research directions.

\chapter{Literature Review}
Probabilistic forecasting has emerged as an essential tool for managing uncertainty across multiple domains—electricity price forecasting, wind power forecasting, stock price prediction, and web traffic forecasting. Classic point forecasting techniques often do not and cannot, by their very nature, fully capture the intrinsic variability at play in these fields, making probabilistic approaches more appealing to risk-sensitive decision-making. In the area of electricity price forecasting, models that deliver probability distributions of prospective prices have become increasingly popular. Two popular probabilistic forecasting methods, namely the LUBE (Lower Upper Bound Estimation) method and QR (Quantile Regression) based method, are proposed by \cite{12} and \cite{13}, which have given rise to various other methods that have incorporated these two methods. A combined probabilistic forecasting system, CPFS, synergizing quantile regression with neural networks, has established its superiority over singular model methodologies in both interval width and coverage probability and, therefore, given more reliable forecasts \cite{9}. Similarly, Gaussian Processes have been used in wind power forecasting to help in modeling the interaction that might exist between wind power generation and meteorological conditions. \cite{10} presents a multitask Gaussian process (MTGP) method successfully applied to scenarios with scarce historical data by transferring knowledge from similar source tasks to enhance the predictions of new wind farms.

In the case of prediction for stock prices, deep learning architecture-based models such as LSTM, GRU, and CNN can effectively capture the complexity and non-linearity observed within financial markets. Other models are uncertain, and there is a need to use probabilistic methods such as QR and KDE to increase the reliability of the forecast \cite{4, 9}. Deep learning architectures, particularly Long Short-Term Memory (LSTM) networks and Gated Recurrent Units (GRU), have exhibited the ability to encompass the subtle complexities inherent in the data; however, their effectiveness can be enhanced through the use of non-parametric approaches, providing better coverage intervals and better decision-making capabilities \cite{8, 3}. The application area also includes web traffic prediction, sharing patterns with similar spiky erratic changes in stock price forecasting, which also benefits from the combination of machine learning techniques. Methods such as SVM, with the incorporation of ensemble methods, have been employed to predict web traffic, thereby allowing for the allocation of resources and strategic planning of websites within precise prediction intervals \cite{5, 7}. In addition, swarm intelligence methods, including PSO and WOA, have successfully been implemented for hybrid models in forecasting; this has improved the precision of predictions in dynamic conditions \cite{9}. The key development in this case is the implementation of hybrid models integrating probabilistic forecasting with machine learning and optimization techniques, making the establishment of higher accuracy of forecasting within these fields more achievable and providing greater resilience to data uncertainty \cite{1, 6}. In conclusion, probabilistic forecasting remains fundamental in the fight against challenges attached to volatility in industries such as energy, finance, and online services, wherein uncertainty and nonlinearity are integral to the decision-making process \cite{2, 10, 9}.

Recent advancements further underscore the potential of hybrid models. For instance, \cite{14} discuss blending climate predictions with AI models to enhance hydroclimatic variable predictability, emphasizing the benefits of combining dynamical and data-driven models. Similarly, \cite{15} propose two-stage hybrid models to address heterogeneity in time series data, demonstrating improved forecasting performance by capturing both global and local patterns. In the realm of wind power forecasting, \cite{16} introduces an adaptive quantile regression approach resilient to missing values, highlighting the importance of robustness in probabilistic models. Additionally, \cite{17} present a probabilistic forecasting method for regional solar power that incorporates weather pattern diversity, showcasing the adaptability of hybrid models to various energy domains.

Despite these advancements, several research gaps remain. Most existing methods struggle to simultaneously ensure high prediction interval coverage probability (PICP) and maintain narrow prediction intervals (PINAW). For instance, QR-based methods often produce precise but under-confident forecasts, while LUBE-based methods may offer good coverage but at the cost of excessively wide intervals. There is also a lack of standardized benchmarking across different confidence levels and datasets, making it difficult to generalize findings. Furthermore, many methods are either purely parametric or purely data-driven, limiting their robustness in real-world applications where noise and variance are high.

These limitations provide a strong motivation for the development of novel hybrid methods. By combining the strengths of LUBE-based interval estimation with statistical or deep learning-based enhancements, it is possible to strike a better balance between reliability and efficiency. For instance, hybridizing LUBE with Quantile Regression (QR) introduces complementary strengths like LUBE's high coverage with QR’s narrower interval width. Similarly, integrating LUBE with Parametric models such as the Weibull distribution enables modeling of residual uncertainty and tail behavior more effectively, improving coverage in volatile scenarios. These hybrid approaches are designed to adapt across multiple model architectures (e.g., LSTM, GRU, CNN, BiLSTM) and datasets (e.g., stock prices, electricity load, web traffic), ensuring more generalizable and practically applicable forecasting systems. Thus, the present work addresses these gaps by proposing and evaluating such hybrid probabilistic forecasting frameworks that offer robustness, adaptability, and improved decision-support capacity in uncertain environments.

\begin{table}[!ht]
    \caption{Recent Literature on Probabilistic Time Series Forecasting Methods.} \vspace{0.2cm}
    \centering
    \resizebox{\textwidth}{!}{ 
    \begin{tabular}{|p{0.5cm}|p{2.8cm}|p{2cm}|p{2cm}|p{3.2cm}|p{3.2cm}|p{2.5cm}|p{2.5cm}|}
        \hline
        \textbf{Id} & \textbf{Reference} & \textbf{Model} & \textbf{Method} & \textbf{Merits} & \textbf{Drawbacks} & \textbf{Dataset} & \textbf{Accuracy Measure} \\ \hline
        \cite{12} & Abbas Khosravi et al. (2011), IEEE transactions on Neural Networks & Neural Networks & LUBE (Lower-Upper Bound Estimation) & Produces reliable prediction intervals without assuming noise distribution. Applicable to various forecasting problems. & Tuning of custom loss functions is complex. Training instability may occur. & Benchmark time series datasets & PICP, PINAW \\ \hline
        \cite{2} & Cameron Cornell et al. (2024), Int. Journal of Forecasting & Neural Networks & Probabilistic Forecasting for Electricity Prices & Adapts well to market volatility, integrates market mechanisms in forecasting. & Model sensitivity to sudden price spikes and data availability issues. & Australian National Electricity Market (NEM) & CRPS, RMSE, PICP \\ \hline
        \cite{9} & Yan Xu et al. (2024), Computers and Industrial Engineering & Quantile Regression & Quantile Combination Forecasting & Combines multiple quantile forecasts to improve robustness and accuracy. & Complexity in selecting and weighting contributing models. & Electricity price datasets & PICP, Quantile Loss \\ \hline
        \cite{1} & Sourav Kumar Purohit and S. Panigrahi (2024), Information Sciences & Hybrid (Statistical + DL) & Optimized Deep Learning & Integrates classical and deep models for improved oil price forecasting. & High computational demand; interpretability challenges. & Crude oil market datasets & RMSE, MAE, PICP \\ \hline
        \cite{3} & Hao-Hsuan Huang and Yun-Hsun Huang (2024), Energy Reports & Deep Learning & Probabilistic Forecasting with Weather Diversity & Accounts for weather diversity in regional solar forecasting. Enhances generalization. & Relies heavily on accurate meteorological inputs. & Regional solar power datasets & PICP, PINAW, RMSE \\ \hline
    \end{tabular}
    }
    \label{tab:consolidated_lit_review}
\end{table}


Table ~\ref{tab:consolidated_lit_review} summarizes recent literature on probabilistic time series forecasting methods. It highlights key studies involving different models and approaches, such as Neural Networks with LUBE, Quantile Regression, Hybrid statistical-deep learning methods and deep learning techniques tailored for specific applications like electricity price and solar power forecasting. For each reference, the table outlines the main merits and drawbacks, datasets used, and accuracy measures employed. This consolidated view provides insight into the strengths and limitations of various probabilistic forecasting methods across diverse domains, guiding future research directions.

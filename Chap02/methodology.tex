\chapter{Proposed Methodology and System Design} \label{Methodology} 

\section{Motivation}
Nowadays, healthcare systems are overwhelmed with patients who have to be monitored continuously because of their chronic conditions, postoperative situations, or remote living. Hospitals equipped with conventional monitoring devices are not only time-consuming and costly but also impractical for long-term monitoring. Given the developments in IoT and cloud computing, Remote Patient Monitoring (RPM) offers the possibility of real-time health data acquisition from patients and direct data transfer to doctors or hospitals. On the other hand, the biggest issue that security and privacy concerns regarding the medical data being transmitted.

Most of the present IoT-based solutions have either password-based or RSA-based authentication mechanisms, which can be weak or computationally expensive for small devices such as Arduino or ESP modules. Hence, this project is arisen from the demand for a healthcare data privacy ensuring, third-party access denial preventing, and patient authentication process confidential and secure yet lightweight, resistant to the security threats, and reliable in performance problem.

Therefore, the solution is using the:
\begin{itemize}
   \item Elliptic Curve Cryptography (ECC) for light encryption,
   \item Record management via blockchain technology, which is very secure since it is tamper-proof, and
   \item A cloud-based system to facilitate easy access in real time by the doctors.
\end{itemize}

\section{System Architecture}
Proposed system is a combination of IoT devices, cloud storages, and security based on blockchain. It can be divided into four primary layers:

\subsection{Patient Layer (Device Layer)}
\begin{itemize}
   \item Temperature Sensor (LM35) and Pulse Oximeter (MAX30100) are examples of the sensors that might be included in this system.
   \item These sensors are linked to an Arduino UNO that gathers real-time health data of the patient.
   \item The encrypted data is sent to the cloud by an ESP8266 Wi-Fi module.
\end{itemize}

\subsection{Network / Cloud Layer}
\begin{itemize}
   \item The ThingSpeak IoT Cloud Platform is put to use for storing and receiving data coming from a patient.
   \item The information is in an encrypted format, which uses the session key created via ECC.

Before accepting the data, the blockchain verifies the sender (patient’s device) to be genuine.

\end{itemize}

\subsection{Blockchain Authentication Layer}
\begin{itemize}
\item Every patient device enrolled is given a digital smart card that contains authentication credentials.

\item These smartcards are written in the blockchain ledger, which makes them very secure and cannot be altered easily.

\item During each connection, the mutual authentication process occurs between:
\begin{itemize}
\item Patient device,
\item Cloud server, and
\item Hospital server.
\end{itemize}

\item Patient device,

\item Cloud server, and

\item Hospital server.

\item None of the devices will be given their connecting request if the verification is not successful.
\end{itemize}

\subsection{Hospital Server and Doctor Layer}
\begin{itemize}
\item The Hospital Server through a secure dashboard validates the information arriving from the blockchain.

\item Doctors access the patient’s condition in real-time through the server and intervene if they find any abnormality.
\end{itemize}

\section{Working Flow}
\subsection{Patient Registration}
The patient’s device is registered with the hospital server. A smart card with cryptographic parameters is generated and saved on the blockchain.

\subsection{Data Collection}
The sensors take the vital signs of the patient (temperature, SpO₂) and forward the data to the Arduino.

\subsection{Data Encryption}
ECC is implemented by Arduino to encode the data with the session key (SK).

\subsection{Data Transmission}
Data in encrypted form is being transferred to the ThingSpeak Cloud through Wi-Fi.

\subsection{Authentication and Verification}
A blockchain performs a check between the device credentials and the records. If the verification is positive, the data is allowed.

\subsection{Data Access by Doctor}
The data, after verification, is pulled out by the hospital server, and the doctor is able to follow up with the patient from a distance.



\section{Algorithm Explanation}
The presented scheme is a simplified, lightweight, blockchain-enabled authentication version from the Internet of Drones paper, that was the basis for the system.

It has three steps:

\subsection{Phase 1: Registration}
\begin{itemize}
    \item The hospital server sets up elliptic curve parameters.
    \item Patient device forwards identity and password to the server.
    \item Server prepares a smart card and puts its hash on the blockchain.
    \item The patient gets the smart card that encloses public parameters.
\end{itemize}

\subsection{Phase 2: Authentication}
\begin{itemize}
    \item Smart card is used by the patient to log in.
    \item Nonces are generated on both sides, ECC is performed to get the session key (SK) operation.
    \item Ethereum smart contracts validate identities and control impersonation.
    \item If the operation succeeds, the parties get a secure channel.
\end{itemize}

\subsection{Phase 3: Data Transmission}
\begin{itemize}
    \item Patient data is encrypted by the Arduino with the session key.
    \item Data are securely transmitted to the cloud.
    \item Hospital server gets the data decrypted for doctors' viewing.
\end{itemize}

\section{Mathematical Overview of ECC (Simplified)}
The equation for the elliptic curve cryptography (ECC) curve is:

\[
    y^2 = x^3 + a x + b \ (\text{mod } p)
\]

Every party has:
\begin{itemize}
    \item A private key (d)
    \item A public key (Q = d \times G), where G is a point on the curve.
    \item A private key (d)
    \item A public key (Q = d \times G), where G is a point on the curve.
\end{itemize}

The session key (SK) is generated by:

\[
    SK = (d_A \times Q_B) = (d_B \times Q_A)
\]

which allows both parties to have the same shared key without letting other parties know it.

This is the key that is utilized for encrypting the medical data before the data is sent to the cloud.

\begin{table}[!ht]
    \centering
    \caption{System Stages, Components, and Purpose}
    \begin{tabular}{|p{3cm}|p{6cm}|p{6cm}|}
        \hline
        	extbf{Stage} & \textbf{Component Used} & \textbf{Purpose} \\ \hline
        Data Collection & Temperature \& SpO₂ Sensors, Arduino UNO & Capture real-time health data \\ \hline
        Transmission & ESP8266 Wi-Fi Module & Send data to cloud securely \\ \hline
        Authentication & Blockchain + ECC & Verify device and user identity \\ \hline
        Data Storage & ThingSpeak Cloud & Store encrypted patient data \\ \hline
        Data Access & Hospital Server, Doctor Interface & View and monitor health status \\ \hline
    \end{tabular}
    \label{tab:system_stages}
\end{table}


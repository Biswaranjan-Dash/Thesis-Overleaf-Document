\thispagestyle{empty} 
\SpecialTitle{Abstract}
\noindent This project proposes a secure remote patient monitoring system design and its subsequent implementation employing a lightweight blockchain-enabled authentication approach. The described system keeps on recording patients' health parameters like body temperature, oxygen saturation, and so on, through biomedical sensors connected to an Arduino UNO and an ESP Wi-Fi module. After the data collection, it is stored in a cloud platform (ThingSpeak) securely for the purpose of real-time storage and analysis.\\
The use of a blockchain-based authentication scheme along with the use of Elliptic Curve Cryptography (ECC) is a method that guarantees the fact that only the devices that have been verified and hospital servers have the capability to access or transmit sensitive medical data. To this end, the integrity of the data, its confidentiality, and the privacy of the user are improved while simultaneously, the computational costs are kept low which is very important for resource-constrained IoT devices. After that the hospital server fetches the legitimate data from the cloud, thus the doctors will be able to monitor the patients remotely and make the necessary clinical decisions timely.\\
The proposed system demonstrates an efficient, scalable, and secure solution for healthcare IoT applications, effectively preventing common network attacks such as replay, impersonation, and man-in-the-middle attacks.

\vspace{5mm}
\noindent\textit\textbf{Keywords:} Remote Patient Monitoring, Internet of Medical Things (IoMT), Blockchain, Lightweight Authentication, Elliptic Curve Cryptography (ECC), Secure Healthcare System.